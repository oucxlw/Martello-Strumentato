\chapter{Conclusioni e sviluppi futuri}
\label{chap:cap5}
Gli IMBHs, la loro esistenza ed il fatto di poterli trovare al centro dei GCs sono stati per lungo tempo oggetto di dibattito nel mondo scientifico.
I modelli teorici attuali indicano che è possibile identificarli negli ammassi globulari, mentre i metodi osservativi usati finora non ne hanno ancora verificato la presenza. Un interessante metodo proposto per identificarli è quello che si basa sullo studio dinamico delle MSPs in ammasso. Da qui nasce l'idea di questo lavoro di tesi: voler identificare gli IMBHs in ammassi globulari attraverso una tecnica di ML interpretabile utilizzando le caratteristiche dinamiche delle MSPs. Si tratta, infatti, di voler applicare un metodo nuovo ad un problema ben noto.

A partire dai risultati di un set di simulazioni a N-corpi di ammassi globulari sono stati creati i \textit{dataset} forniti all'algoritmo di ML. Lo scopo è stato quello di voler prevedere la presenza di IMBHs in questi ammassi sulla base della posizione nel piano del cielo delle MSPs e delle loro componenti lungo la linea di vista di velocità, accelerazione, \textit{jerk} e \textit{snap}. Per farlo ci si è serviti degli alberi decisionali.\\
Sono stati considerati circa 200 ammassi, per ognuno dei quali sono state effettuate 6 estrazioni casuali di un numero variabile tra 1 e 40 di MSPs selezionate casualmente dalle regioni centrali degli ammassi stessi. 

Nonostante gli alberi decisionali non siano tra i modelli di ML più sofisticati, si ottengono dei buoni risultati. In particolare, si riescono a classificare gli ammassi in base alla presenza di un IMBH con valori di accuratezza pari a circa il $70\%$.\\ 
Inoltre, la caratteristica fondamentale degli alberi decisionali è che essi possono essere facilmente interpretati, specie quando si tratta di alberi non molto profondi. Per questo motivo, questi modelli vengono spesso preferiti rispetto ad altri.\\
Dall'interpretazione delle scelte che il modello compie in ogni nodo dell'albero è possibile capire quali sono state le caratteristiche dinamiche delle MSPs che hanno avuto più importanza nella fase di allenamento del modello. Per tutti i casi analizzati risulta che, come atteso, l'importanza maggiore viene attribuita alle accelerazioni delle MSPs negli ammassi. Molto importante risulta anche il raggio proietatto sul piano del cielo della MSP più distante. Il resto delle importanze è distribuito tra gli \textit{snaps} ed i raggi proiettati delle altre MSPs.

Nella pratica, dopo aver monitorato un certo numero di MSPs in un ammasso misurandone le caratteristiche dinamiche, quello che si vuole capire è se, in base ai loro valori, l'ammasso possa ospitare un IMBH o meno. Se in questa fase avessimo già allenato un modello di ML a partire da dati di sumilazioni, allora si potrebbe prevedere nella realtà la presenza di tali oggetti tanto ricercati.\\ L'importanza di questo lavoro sta proprio nel fatto che sarebbe possibile ottenere un riscontro futuro su dati veri. Per tale ragione, sarebbe fondamentale affinare questo modello e svilupparne altri a partire da questo, per ottenere previsioni più accurate.\\
Un primo passo verso il miglioramento delle \textit{performance} dell'algoritmo è quello di rendere le simulazioni da cui provengono i \textit{dataset} il più verosimili possibile, ad esempio considerando anche l'evoluzione stellare. In secondo luogo, invece, si potrebbero migliorare alcuni punti importanti del codice di previsione. Infatti, il criterio considerato per determinare la presenza di un IMBH in un ammasso si basa su un valore di soglia di massa relativa scelto a priori. Questo passaggio è cruciale per la classificazione degli ammassi nelle due distinte classi, per questo sarebbe importante svolgere, in futuro, uno studio più approfondito sulla scelta di tale criterio (\textbf{suggerimenti su un possibile metodo?}).

Infine, è lecito sostenere che l'originalità di questo studio stia proprio nei metodi applicati ad una questione astrofisica ormai nota. Per questo sarebbe importante concentrarsi sui suoi possibili sviluppi e miglioramenti, con la prospettiva che possa essere utile per ricerche future.
