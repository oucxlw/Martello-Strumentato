\chapter*{Introduzione}
\addcontentsline{toc}{chapter}{Introduzione}
\label{chap:intro}
I buchi neri di massa intermedia (\textit{Intermediate-Mass Black Holes}, IMBHs) sono buchi neri la cui massa è compresa tra $10^{2}$ e $10^{6}M_{\odot}$. Si pensa che essi possano essere l'anello mancante necessario per collegare le informazioni note riguardanti i buchi neri di origine stellare ($\sim 10 M_{\odot}$) a quelle riguardanti i buchi neri super-massicci ($10^{6}-10^{10}M_{\odot}$). Infatti, i modelli teorici prevedono l'esistenza degli IMBHs al centro di ammassi globulari (\textit{Globular Cluster}, GC), mentre le osservazioni condotte finora non ne hanno ancora rilevato la presenza \MP{a parte il recentissimo evento di detection di onde gravitazionali GW190521, che viene intepretato come la formazione di un buco nero di $\approx 150$ masse solari \cite{Abbott2020:paper}, quindi in ogni caso di massa abbastanza vicina all'estremo stellare del range di massa degli IMBH}. 

La motivazione per cui questi oggetti peculiari risultano sfuggenti rispetto ai buchi neri appartenenti alle altre categorie è legata principalmente al fatto che la strumentazione attuale presenti dei limiti in tal senso.  
Tuttavia, esistono dei metodi indiretti per poter identificare gli IMBHs negli ammassi. Uno fra essi è quello che si basa sulla dinamica delle MSPs nei GCs. In particolare, attraverso misure delle derivate dei periodi delle MSPs sarebbe possibile ottenere i loro valori di accelerazione e delle sue derivate: il \textit{jerk} (derivata prima dell'accelerazione) e lo \textit{snap} (derivata seconda dell'accelerazione). Dallo studio di tali caratteristiche, poi, si potrebbe risalire all'identificazione degli IMBHs \cite{abbate1:paper}. 

A partire da questa proposta, questo lavoro si sviluppa verso tale scopo, ma utilizzando metodi diversi. Infatti, vengono utilizzate tecniche di \textit{Machine Learning} interpretabile per prevedere la presenza di IMBHs al centro di GCs. 

Poiché i dati osservativi in questo campo non sono sufficienti, per lo studio si sono utilizzati dati ottenuti da un set di simulazioni a N-corpi. Il modello di \textit{Machine Learning} utilizzato è quello degli alberi decisionali, caratterizzati dalla possibilità di essere interpretati. Questa è un' importante caratteristica per un modello previsionale che riguarda soprattutto la fase di analisi dei risultati. 

Gli ammassi simulati sono stati classificati dal modello secondo la presenza di un IMBH o meno sulla base delle caratteristiche dinamiche delle MSPs scelte nelle regioni centrali dei GCs. Nello specifico, le caratteristiche fornite al modello sono: la distanza delle MSPs dal centro di massa del GC proiettata sul piano del cielo e le componenti di velocità, accelerazione, \textit{jerk} e \textit{snap}, considerate lungo la linea di vista. 

La ricerca di IMBHs attraverso questo metodo potrebbe essere una nuova strada alternativa da percorrere per ottenere dei risultati soddisfacienti, specie se tali metodi in futuro potranno essere applicati a dati reali.

Il lavoro di tesi può essere suddiviso principalmente in tre parti. La prima e la seconda parte  si sviluppano rispettivamente nei capitoli \ref{chap:cap1} e \ref{chap:cap2}. La terza parte, invece, contiene i restanti tre capitoli.\\
Nel primo capitolo viene presentato e contestualizzato il problema astrofisico su cui si basa l'intera tesi: la ricerca degli IMBHs in ammassi globulari. Viene, dunque, fornita un'introduzione sugli IMBHs e su quali sono i metodi utilizzati finora per identificarli, per poi concentrarsi, in particolare, sul metodo delle MSPs.\\
Nel secondo capitolo, invece, viene descritto lo strumento utilizzato per lo sviluppo del progetto, ovvero gli alberi decisionali. Dopo una breve introduzione al \textit{Machine Learning}, viene descritto l'algoritmo CART \cite{brei:book} che è quello utilizzato per la costruzione degli alberi decisionali in questo lavoro. Vengono, infatti, descritte le fasi in cui l'algoritmo si sviluppa e le metodologie che esso adopera, come, ad esempio, la tecnica del \textit{pruning} utilizzata in questo caso per risolvere un problema comune a tutti gli algoritmi di \textit{Machine Learning}: il problema dell'\textit{overfitting}. Inoltre, in questo capitolo, vengono forniti gli strumenti per valutare le prestazioni di un algoritmo di classificazione.\\
Infine, nell'ultima parte si entra nel merito del lavoro di tesi descrivendo nello specifico i \textit{dataset} utilizzati, il codice sviluppato per la classificazione degli ammassi (Cap. \ref{chap:cap3}) e, in ultimo, vengono presentati i risultati ottenuti con la relativa interpretazione (Cap. \ref{chap:cap4}). Nel quinto ed ultimo capitolo, invece, vengono discusse le conclusioni e presentate le possibili prospettive future per l'identificazione di IMBHs in GCs attraverso tecniche di \textit{Machine Learning}.






